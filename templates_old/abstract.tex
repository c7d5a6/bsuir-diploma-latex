\sectioncentered*{Реферат}
\thispagestyle{empty}

\emph{Ключевые слова}: вероятностные модели; байесовы сети; вывод структуры сети по данным; принцип минимальной длинны описания; оценка апостериорной вероятности.

\vspace{4\parsep}

Дипломный проект выполнен на 6 листах формата А1 с пояснительной запиской на~\pageref*{LastPage} страницах, без приложений справочного или информационного характера. 
Пояснительная записка включает \total{section}~глав, \totfig{}~рисунков, \tottab{}~таблиц, \toteq{}~формул и \totref{}~литературный источник.

Целью дипломного проекта является разработка удобного в использовании инструмента, пригодного для решения практических задач, возникающих в реальных проектах, связанных с вероятностным моделированием.

Для достижения цели дипломного проекта была разработана библиотека кода для \dotnet{}, предназначенная для представления и обучения структуры вероятностной сети по экспериментальным данным.
Библиотека может быть использована в реальных проектах, использующих вероятностный подход к решению проблемы.
В библиотеке реализовано несколько алгоритмов, имеющих различные качественные характеристики.

В разделе технико"=экономического обоснования был произведён расчёт затрат на создание ПО, а также прибыли от разработки, получаемой разработчиком.
Проведённые расчёты показали экономическую целесообразность проекта.

Пояснительная записка включает раздел по охране труда, в котором была произведена оценка пожарной безопасности на предприятии, где частично разрабатывался данный дипломный проект.

\clearpage